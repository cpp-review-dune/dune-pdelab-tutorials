\documentclass[12pt,a4paper]{article}

\usepackage[utf8]{inputenc}
\usepackage[a4paper,total={150mm,240mm}]{geometry}
\usepackage[american]{babel}

\usepackage{float}
\usepackage{babel}
\usepackage{amsmath}
\usepackage{tikz}
\usepackage{graphicx}
\usepackage{amssymb}
\usepackage{todonotes}


\usepackage{listings}
\definecolor{listingbg}{gray}{0.95}
\lstset{language=C++,basicstyle=\ttfamily\small,frame=single,backgroundcolor=\color{listingbg}}
% \lstset{language=C++, basicstyle=\ttfamily,
%   keywordstyle=\color{black}\bfseries, tabsize=4, stringstyle=\ttfamily,
%   commentstyle=\it, extendedchars=true, escapeinside={/*@}{@*/}}


\newcommand{\vx}{\vec x}
\newcommand{\grad}{\vec \nabla}
\newcommand{\wind}{\vec \beta}
\newcommand{\Laplace}{\Delta}
\newcommand{\mycomment}[1]{}


% Exercise stylesheet
\usepackage{exercise}

\title{\textbf{Exercises for Tutorial01}}
\subtitle{TODO Subtitle?}
\exerciselabel{Exercise}




\begin{document}

\exerciseheader

\begin{Exercise}{TODO NAME}
  Would make sense to have an exercise where they play around with the
  code. Easy:
  \begin{itemize}
    \item Grids
    \item degree
    \item dimension
    \item Convergence order would be possible if they adapt $f$ in such a way
      that $g$ is the solution.
    \item Right now the code only uses dirichlet boundary.
    \end{itemize}
\end{Exercise}

\begin{Exercise}{Nitsche's method for weak Dirichlet boundary
    conditions}
  In this exercise we want to implement Dirichlet boundary conditions
  in a weak sense by using Nitsche's method.  The residual takes the
  form:

  \begin{align*}
    r^{\text{Nitsche}}(u,v) &= \int_\Omega \nabla u \cdot \nabla v + (q(u)-f)v\,dx + \int_{\Gamma_N} jv\,ds \\
    &\quad - \int_{\Gamma_D} \nabla u \cdot\nu v\,ds - \int_{\Gamma_D} (u-g)\nabla v \cdot\nu\,ds
    + \eta \int_{\Gamma_D} (u-g)v\,ds
  \end{align*}

  In order to implement these changes we need to do the following:
  \begin{itemize}
  \item Adapt constraints.
  \item Adapt LOP.
  \end{itemize}


\end{Exercise}


\end{document}
