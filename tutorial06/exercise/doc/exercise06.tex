\documentclass[12pt,a4paper]{article}

\usepackage[utf8]{inputenc}
\usepackage[a4paper,total={150mm,240mm}]{geometry}
\usepackage[american]{babel}

\usepackage{float}
\usepackage{babel}
\usepackage{amsmath}
\usepackage{tikz}
\usepackage{graphicx}
\usepackage{amssymb}
\usepackage{todonotes}
\usepackage{hyperref}


\usepackage{listings}
\definecolor{listingbg}{gray}{0.95}
\lstset{language=C++,basicstyle=\ttfamily\small,frame=single,backgroundcolor=\color{listingbg}}
% \lstset{language=C++, basicstyle=\ttfamily,
%   keywordstyle=\color{black}\bfseries, tabsize=4, stringstyle=\ttfamily,
%   commentstyle=\it, extendedchars=true, escapeinside={/*@}{@*/}}


\newcommand{\vx}{\vec x}
\newcommand{\grad}{\vec \nabla}
\newcommand{\wind}{\vec \beta}
\newcommand{\Laplace}{\Delta}
\newcommand{\mycomment}[1]{}


% Exercise stylesheet
\usepackage{exercise}

\title{\textbf{Exercises for Tutorial06}}
\subtitle{Parallelism}
\exerciselabel{Exercise}


\begin{document}

\exerciseheader

This exercise will show you two things:
\begin{enumerate}
\item For common problems it is really easy to run your code in
  parallel. Almost everything is done by PDELab and you only have to make
  some small changes compared to the sequential code.
\item If you have a more complicated problem this exercise will give
  you some hints how to start implementing it.
\end{enumerate}

Some remarks regarding parallel programs in DUNE:
\begin{itemize}
\item Parallel DUNE programs utilize the \emph{Message Passing
    Interface} MPI. Many different MPI implementations are known to
  work well with DUNE.
\item By default CMake builds your programs for parallel use so you
  can build the exercises for this tutorial the same way as usual.
\item To actually run the parallel program, you need to call the
  \lstinline{mpirun} command of your MPI implementation. If you want
  to run exercise06-1 with two processes you could use
  \begin{lstlisting}
mpirun --oversubscribe -np 2 ./exercise06-1
  \end{lstlisting}
  Note that \lstinline{--oversubscribe} may be necessary if the number of processes you want to start exceeds the number of your physical processor cores.
\item VTK output from \lstinline{VTKWriter} does not need any special
  treatment in the parallel case. When applied to a
  \lstinline{GridView} of a parallel grid each process will produce
  output corresponding to his local sub grid (\lstinline{*.vtu}
  files).  Furthermore the process of zero id will create a file
  (\lstinline{*.pvtu}) which corresponds to the global grid.
\end{itemize}

\begin{Exercise}{Running the Code from the Tutorial}
  In this exercise you will run the code from the tutorial in parallel
  and make some simple changes.
  \begin{enumerate}
  \item First run the code from \lstinline{exercise06-1} and try out
    different settings. Don't forget to use \lstinline{mpirun} if you
    want to use more than one process.
  \item Right now the program calculates the squared L2 error separately on
    every process without communicating. Use
    \lstinline{CollectiveCommunication}
    \footnote{\href{https://www.dune-project.org/doxygen/pdelab/master/classDune\_1\_1CollectiveCommunication.html}{https://www.dune-project.org/doxygen/pdelab/master/classDune\_1\_1CollectiveCommunication.html}}
    to calculate the sum over the local L2 errors (squared).
  \item You could also try different solver backends.
  \end{enumerate}
\end{Exercise}

\begin{Exercise}{Communication over the Grid}
  In this exercise you will learn how to communicate data using overlapping
  grids and a \lstinline{DataHandle}. This can be useful if you need to write a
  parallel application and if you need to do something that is not covered by
  the \lstinline{CollectiveCommunication} class.

  Communication in the overlap happens entity-wise. That means you have entity
  related data and each entity in the overlap can send this data to the
  corresponding entity of the other processor. You can control which
  codim-entities should send data and for which part of the overlap
  communication will happen (see partition types in the slides for tutorial06).

  The source code for this exercise can be found in the file
  \lstinline{exercise06-2.cc}. Right now the code does the following:
  \begin{itemize}
  \item It creates an overlapping \lstinline{YaspGrid} in two
    dimensions. You can set the overlap in the ini file
    \lstinline{datahandle.ini}.
  \item The communication happens in the file
    \lstinline{communication.hh}.  We create the vector
    \lstinline{std::vector<int> data} that stores one int for every
    entity of a given codimension. If you for example choose
    \lstinline{codim=1} in the ini file the data vector will have one
    entry for every edge of the current process.
  \item We set all entries of the data vector equal to the rank of the
    current process, i.e. on rank zero all entries will be zero on
    rank one all entries will be one and so on.
  \item With
    \begin{lstlisting}
// Data handle to communicate the data
// of each edge to the next processor.
using DH = ExampleDataHandle<IndexSet, decltype(data)>;
DH dh(indexSet, data, cdim);
    \end{lstlisting}
    we create a data handle. The class \lstinline{ExampleDataHandle}
    is defined above.
  \item Communication happens by calling the \lstinline{communicate}
    method of the \lstinline{GridView}:
    \begin{lstlisting}
gv.communicate(dh,Dune::All_All_Interface,
               Dune::ForwardCommunication);
    \end{lstlisting}
    This methods needs a \lstinline{DataHandle}, a communication
    \lstinline{InterfaceType} and a
    \lstinline{CommunicationDirection}. More information about these
    can be found in the documentation
    \footnote{\href{http://www.dune-project.org/doc-pdelab-master/doxygen/html/group\_\_GIRelatedTypes.html}{http://www.dune-project.org/doc-pdelab-master/doxygen/html/group\_\_GIRelatedTypes.html}}
    but we don't need that for completing the exercise.
  \end{itemize}
  Our goal is to communicate the rank stored in the data vector to the
  other process in the overlap regions. Right now the
  \lstinline{ExampleDataHandle} on top of the file does not
  communicate anything. Let's change that!

  \begin{enumerate}
  \item Remove the comments from these lines:
    \begin{lstlisting}
// switch (communicationType){
// case 1: gv.communicate(...); break;
// case 2: gv.communicate(...); break;
// case 3: gv.communicate(...); break;
// case 4: gv.communicate(...); break;
// default: gv.communicate(...);
// }
    \end{lstlisting}
  \item All methods the \lstinline{DataHandle} needs are already there
    but the implementations are missing.  With the comments from the
    source code it shouldn't be too hard to fill in the missing
    peaces.
  \item After implementing the missing parts or copying from the
    solution try to figure out what the communication
    \lstinline{InterfaceType} means. Run your program with two
    processes:
    \begin{lstlisting}
mpirun --oversubscribe -np 2 ./exercise06-2
    \end{lstlisting}
    Try different communication interfaces by changing the value of
    \lstinline{type} in the ini file:
    \begin{lstlisting}
[communication]
type = 5
    \end{lstlisting}
    Find out where communication happens for different communication
    interfaces by looking at the program output and the comments in
    the source code. Drawing the grid with overlap is highly
    recommended. When you know what's going on try different
    codimensions or larger grids with more overlap.
  \end{enumerate}
\end{Exercise}


\end{document}
